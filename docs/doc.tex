\documentclass[a4paper]{article}
\usepackage[dutch]{babel}
\usepackage{hyperref}
\usepackage{parskip}

\title{Documentatie Spacegame}
\author{Stef Pletinck \and Marieke Louage}

\begin{document}
\maketitle

\section{Debug}
\subsection{Selftest}
Na het maken van alle verbindingen kan het programma deze testen. Deze
functionaliteit is enkel beschikbaar als tijdens het compileren \texttt{DEBUG}
gedefinieerd was, bijvoorbeeld via \texttt{\#{}define DEBUG}.

Houdt tijdens het opstarten van de microcontroller de middelste knop ingedrukt,
op het LCD-scherm zal \emph{Zelftest} verschijnen. Vervolgens zullen de LED's de
kleuren wit, blauw, groen en rood overlopen, terwijl deze kleur op het scherm
wordt aangegeven. Vervolgens kunnen de joysticks getest worden, de
microcontroller zal voor elke joystick vragen om elke richting uit te testen en
zal aangeven wanneer de juiste input gegeven wordt.

Na het uitvoeren van een zelftest kan de microcontroller gereset worden met de
\textsc{reset}-knop om het spel te starten.

\subsection{Tick-by-tick}
De automatische ticks kunnen tijdelijk worden uitgeschakeld door tijdens het
uitvoeren van het spel de middelste knop ingedrukt te houden voor minstens 1
seconde. Vanaf dan zal er bij elke druk op deze knop 1 tick verstrijken. Na
opnieuw de knop ingedrukt houden gaat het spel weer naar normaal. Deze
functionaliteit is enkel beschikbaar als tijdens het compileren \texttt{DEBUG}
gedefinieerd was.

\section{Het spel}
\subsection{Opstarten}
Eerst kan de microcontroller spanning krijgen en opstarten. Van zodra op het
LCD-scherm cijfers verschijnen kan de voeding voor de ESC ingeschakeld worden.
Indien de ESC een continue serie hoge bieptonen afspeelt, is dit mislukt en moet
opnieuw begonnen worden. Indien alles gelukt is, zal na enkele seconden de
schijf langzaam beginnen versnellen (indien nodig kunt u deze een duwtje geven).
Wanneer deze op volle snelheid is zal het LCD-scherm blanco worden en kan
gespeeld worden.

\subsection{Het spel starten}
Na het opstarten is het volledige display groen. Wanneer beide spelers hun
joystick omhoog duwen, start het spel.

\subsection{Bediening}
Spelers kunnen stijgen en dalen door de joystick omhoog of omlaag te duwen. Let
op: wanneer een speler reeds aan de binnen- of buitenkant is en verder in die
richting duwt, zal deze langzaam beginnen te sterven tot er in de andere
richting wordt geduwd.

Spelers kunnen zowel naar links als naar rechts schieten, door de joystick in de
gepaste richting te duwen. Kogels discrimineren niet, ze doen pijn voor
iedereen. Kogels zakken langzaam naar het midden van de aarde, net zoals
spelers. Kogels verdwijnen een tijdje na het afschieten vanzelf. Wanneer er 18
kogels in het spel zijn, kan er niet meer geschoten worden tot er een kogel verdwijnt.

\subsection{Einde van het spel}
Wanneer alle levenskracht van een speler op is, zal het spel eindigen. Het
volledige display krijgt dan de kleur van het schip van de winnaar. Om het spel
te herstarten moet alle voeding uitgeschakeld worden en de hele opstartprocedure
opnieuw gevolgd worden.

\end{document}
