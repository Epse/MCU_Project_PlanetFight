\documentclass{beamer}
\usepackage[dutch]{babel}
\usepackage{graphicx}
\usepackage{siunitx}
\usepackage{underscore}
\usepackage{fontspec}
\usepackage[normalem]{ulem}

\definecolor{ublauw}{HTML}{1E64C8}
\setbeamercolor{title}{fg=ublauw}
\setbeamercolor{subtitle}{fg=black}
\setbeamercolor{frametitle}{fg=ublauw}
\setbeamercolor{framesubtitle}{fg=ublauw}
\setbeamercolor{structure}{fg=ublauw}
\setbeamercolor{section in toc}{fg=ublauw}

\newfontfamily\panno{UGent Panno Text Medium}
\setbeamerfont{title}{family=\fontspec{UGent Panno Text Medium}}
\setbeamerfont{author}{family=\fontspec{UGent Panno Text Medium}}
\setbeamerfont{institute}{family=\fontspec{UGent Panno Text Medium}}
\setbeamerfont{date}{family=\fontspec{UGent Panno Text Medium}}
\setbeamerfont{frametitle}{family=\fontspec{UGent Panno Text Medium}}
\setbeamerfont{framesubtitle}{family=\fontspec{UGent Panno Text Medium}}
\setbeamerfont{section in toc}{family=\fontspec{UGent Panno Text Medium}}
\setbeamerfont{subtitle}{size=\scriptsize,family=\fontspec{UGent Panno Text Medium}}
\setbeamerfont{author}{size=\scriptsize,family=\fontspec{UGent Panno Text Medium}}
\setbeamerfont{institute}{size=\scriptsize,family=\fontspec{UGent Panno Text Medium}}
\setbeamerfont{date}{size=\scriptsize,family=\fontspec{UGent Panno Text Medium}}

\makeatletter
\setbeamertemplate{title page}{
  \vbox{}
  \vfill
  \begingroup
  \centering
  \begin{beamercolorbox}[sep=8pt,center]{title}
    \renewcommand{\ULthickness}{1pt}%
    \usebeamerfont{title}\MakeUppercase{\uline{\inserttitle}}\par%
    \ifx\insertsubtitle\@empty%
    \else%
    \vskip0.25em%
    {\usebeamerfont{subtitle}\usebeamercolor[fg]{subtitle}\insertsubtitle\par}%
    \fi%
  \end{beamercolorbox}%
  \vskip1em\par
  \begin{beamercolorbox}[sep=8pt,center]{author}
    \usebeamerfont{author}\insertauthor
  \end{beamercolorbox}
  \begin{beamercolorbox}[sep=8pt,center]{institute}
    \usebeamerfont{institute}\insertinstitute
  \end{beamercolorbox}
  \begin{beamercolorbox}[sep=8pt,center]{date}
    \usebeamerfont{date}\insertdate
  \end{beamercolorbox}\vskip0.5em
  {\usebeamercolor[fg]{titlegraphic}\inserttitlegraphic\par}
  \endgroup
  \vfill
}
\makeatother

\title{Project Microcontrollers}
\subtitle{Spacegame op POV-Display}
\author{Marieke Louage \and Stef Pletinck}
\institute{UGent Campus Kortrijk}
\date{17 mei 2019}
\logo{\includegraphics[height=2cm]{img/logo.png}}

\begin{document}

\frame{\titlepage}

\begin{frame}
  \frametitle{Inhoud}
  \tableofcontents
\end{frame}

\section{Idee}
\begin{frame}
  \frametitle{Het Idee}
  \begin{itemize}
  \item<1-> Multiplayer spel
  \item<2-> Rond scherm (hardeschijfklok)
  \item<3-> Space Shooter
  \end{itemize}
\end{frame}

\section{Structuur}
\begin{frame}
  \frametitle{Structuur}
  \begin{center}
    \includegraphics[height=0.8\textheight]{img/structuur.jpg}
  \end{center}
\end{frame}

\section{Werking Display}
\begin{frame}
  \frametitle{Werking Display}

  \begin{columns}
    \column{0.5\textwidth}
    \begin{itemize}
    \item Schijf verdeeld in verlichte sectoren
    \item Schijf met gaatjes in spiraal
    \item Optische sensor
    \item Snelle DC-Motor
    \end{itemize}

    \column{0.5\textwidth}
    \includegraphics[width=0.9\textwidth]{img/schijf.jpg}
  \end{columns}
\end{frame}

\section{Motordriver}
\begin{frame}
  \frametitle{Motordriver}

  Een softwaredriver stuurt een ESC\footnote{Electronic Speed Control} module
  aan, die de eigenlijke brushless DC-motor aanstuurt. blabla blablablaa blalbalblaa
\end{frame}

\section{LEDs}
\begin{frame}
  \frametitle{LEDs}

  \begin{columns}
    \column{0.5\textwidth}
    \begin{itemize}
    \item \texttt{APA102} ledstrip
    \item 32 bits per LED
    \item Start- en endframe
    \item SPI
    \end{itemize}

    \column{0.5\textwidth}
    \includegraphics[width=0.9\textwidth]{img/ledpakket.png}
  \end{columns}
\end{frame}

\begin{frame}
  \frametitle{LEDs}
  \framesubtitle{SPI}

  \begin{itemize}
  \item 2 Mogelijkheden: Blocking en interruptgebaseerd
  \item 2 Klokcycli per bit
  \item \SI{68}{\micro\second} hele strip
  \end{itemize}
\end{frame}

\section{Graphics}
\begin{frame}
  \frametitle{Graphics}
\end{frame}

\section{Hoeksnelheidsmeting}
\begin{frame}
  \frametitle{Hoeksnelheidsmeting}
\end{frame}

\section{Game Engine}
\begin{frame}
  \frametitle{Game Engine}

  30 keer per seconde is er een \emph{tick}, ondertussen gebeuren er continu \emph{renders}.
  Er is ook enige debugfunctionaliteit.
\end{frame}

\begin{frame}
  \frametitle{Game Engine}
  \framesubtitle{De Tick}

  \begin{block}{Timing}
    \begin{itemize}
    \item 8-bits Timer/Counter in CTC
    \item \SI{30}{\hertz}
    \item \texttt{maybe_tick()}
    \end{itemize}
  \end{block}

  \begin{block}{Taken}
    \begin{itemize}
    \item Start- en eindscherm
    \item Input
    \item Updates
    \item Botsingen
    \item Test voor einde spel
    \end{itemize}
  \end{block}
\end{frame}

\begin{frame}
  \frametitle{Game Engine}
  \framesubtitle{Render}

  \begin{itemize}
  \item Continu
  \item Aanmaken Sprites
  \item Aansturen \emph{graphics}
  \end{itemize}
\end{frame}

\section{Joysticks}
\begin{frame}
  \frametitle{Joysticks}

  \begin{block}{Fysiek}
    \begin{itemize}
    \item 8 Microswitches
    \item Pullups
    \end{itemize}
  \end{block}

  \begin{block}{Software}
    \begin{itemize}
    \item Uitlezen pins
    \item \texttt{JoyStatus}
    \item Stijgende flanken
    \item Gemaksfuncties
    \end{itemize}
  \end{block}
\end{frame}

\section{Fysieke constructie}
\begin{frame}
  \frametitle{Fysieke constructie}

  \begin{block}{Schijf}
    De draaischijf bestaat uit \SI{3}{\milli\metre} dikke, zwarte ABS,
    uitgesneden op een lasercutter.
  \end{block}

  \begin{block}{Behuizing}
    De behuizing is gelasercut uit MDF, met een plexiglas afdekscherm.
  \end{block}

  \begin{block}{Achterplaat}
    Achter de draaischijf zit een achtergrond van met de hand uitgesneden, witte
    plasticfolie. Deze folie zorgt ook voor schermen tussen de sectoren.
  \end{block}
\end{frame}

\section{Mogelijke verbeteringen}
\begin{frame}
  \frametitle{Mogelijke Verbeteringen}

  \begin{itemize}
    \pause
  \item Een werkend spel (tijdsprobleem)
  \item Display kan vlotter
    \pause
  \item Eigenlijk alles buiten de joysticks en de optische sensor en de
    motordriver en de leddriver
    \pause
  \item Maar de bouwstenen zijn er
  \item Zonder focus op het spel hadden we nu animaties gehad
  \end{itemize}

\end{frame}

\section{Conclusie}
\begin{frame}
  \frametitle{Conclusie}

\end{frame}

\end{document}